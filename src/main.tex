\documentclass{article}
\usepackage{acro}
\usepackage{hyperref}
\usepackage{url}

% Crypto
\DeclareAcronym{PRN}{long = pseudorandom number}
\DeclareAcronym{PRF}{long = pseudorandom function}
\DeclareAcronym{PPT}{long = probabilistic polynomial-time}
\DeclareAcronym{CPA}{long = chosen plaintext attack}

% Security
\DeclareAcronym{TEE}{long = trusted execution environment}
\DeclareAcronym{TCB}{long = trusted computing base}
\DeclareAcronym{SGX}{long = software guard extensions}
\DeclareAcronym{HbC}{long = Honest but Curious}

% Systems
\DeclareAcronym{DAG}{long = directed acyclic graph}
\DeclareAcronym{RDD}{long = resilient distributed dataset}
\DeclareAcronym{UDF}{long = user-defined function}
\DeclareAcronym{BFT}{long = byzantine fault-tolerant}
\DeclareAcronym{SQL}{long = structured query language}
\DeclareAcronym{OLAP}{long = online analytical processing}
\DeclareAcronym{OLTP}{long = online transactional processing}
\DeclareAcronym{DB}{long = database}
\DeclareAcronym{IoT}{long = Internet of Things}
\DeclareAcronym{VM}{long = virtual machine}
\DeclareAcronym{tps}{long = transactions per second}

% Homomorphic Encryption
\DeclareAcronym{FHE}{long = Fully Homomorphic Encryption}
\DeclareAcronym{PHE}{long = Partially Homomorphic Encryption}
\DeclareAcronym{AHE}{long = Additive Homomorphic Encryption}
\DeclareAcronym{MHE}{long = Multiplicative Homomorphic Encryption}
\DeclareAcronym{PPE}{long = Property-Preserving Encryption}
\DeclareAcronym{OPE}{long = Order-Preserving Encryption}
\DeclareAcronym{ORE}{long = Order-Revealing Encryption}
\DeclareAcronym{FPE}{long = Format-Preserving Encryption}
\DeclareAcronym{SRCH}{long = searchable encryption}
\DeclareAcronym{DET}{long = deterministic}
\DeclareAcronym{SDT}{long = Secure Data Type}
\DeclareAcronym{sahe}{
  short = SymAHE,
  long = Symmetric Additive Homomorphic Encryption
}
\DeclareAcronym{smhe}{
  short = SymMHE,
  long = Symmetric Multiplicative Homomorphic Encryption
}
\DeclareAcronym{ASHE}{long = Additively Symmetric Homomorphic Encryption}

% Blockchains
\DeclareAcronym{PoW}{long = proof-of-work}
\DeclareAcronym{PoS}{long = proof-of-stake}



\begin{document}
\title{A personal collection of acronyms using \\ the ``acro'' package}
\author{Savvas Savvides}
\maketitle

\section{How to use}
All acronyms are defined in a file called ``acronyms.tex''. This document shows the most common commands of the acro package. For more information please refer to the official documentation of the Acro package~\cite{acro}.

Make sure you include the file containing the acronyms using the command: \verb=% Crypto
\DeclareAcronym{PRN}{long = pseudorandom number}
\DeclareAcronym{PRF}{long = pseudorandom function}
\DeclareAcronym{PPT}{long = probabilistic polynomial-time}
\DeclareAcronym{CPA}{long = chosen plaintext attack}

% Security
\DeclareAcronym{TEE}{long = trusted execution environment}
\DeclareAcronym{TCB}{long = trusted computing base}
\DeclareAcronym{SGX}{long = software guard extensions}
\DeclareAcronym{HbC}{long = Honest but Curious}

% Systems
\DeclareAcronym{DAG}{long = directed acyclic graph}
\DeclareAcronym{RDD}{long = resilient distributed dataset}
\DeclareAcronym{UDF}{long = user-defined function}
\DeclareAcronym{BFT}{long = byzantine fault-tolerant}
\DeclareAcronym{SQL}{long = structured query language}
\DeclareAcronym{OLAP}{long = online analytical processing}
\DeclareAcronym{OLTP}{long = online transactional processing}
\DeclareAcronym{DB}{long = database}
\DeclareAcronym{IoT}{long = Internet of Things}
\DeclareAcronym{VM}{long = virtual machine}
\DeclareAcronym{tps}{long = transactions per second}

% Homomorphic Encryption
\DeclareAcronym{FHE}{long = Fully Homomorphic Encryption}
\DeclareAcronym{PHE}{long = Partially Homomorphic Encryption}
\DeclareAcronym{AHE}{long = Additive Homomorphic Encryption}
\DeclareAcronym{MHE}{long = Multiplicative Homomorphic Encryption}
\DeclareAcronym{PPE}{long = Property-Preserving Encryption}
\DeclareAcronym{OPE}{long = Order-Preserving Encryption}
\DeclareAcronym{ORE}{long = Order-Revealing Encryption}
\DeclareAcronym{FPE}{long = Format-Preserving Encryption}
\DeclareAcronym{SRCH}{long = searchable encryption}
\DeclareAcronym{DET}{long = deterministic}
\DeclareAcronym{SDT}{long = Secure Data Type}
\DeclareAcronym{sahe}{
  short = SymAHE,
  long = Symmetric Additive Homomorphic Encryption
}
\DeclareAcronym{smhe}{
  short = SymMHE,
  long = Symmetric Multiplicative Homomorphic Encryption
}
\DeclareAcronym{ASHE}{long = Additively Symmetric Homomorphic Encryption}

% Blockchains
\DeclareAcronym{PoW}{long = proof-of-work}
\DeclareAcronym{PoS}{long = proof-of-stake}

=. The most commonly used commands of are:
\begin{description}
  \item{\verb=\acresetall=}
    Resets all acronyms.
  \item{\verb=\ac{}=}
    The most common command. The first occurrence prints the full acronym and its abbreviation in parentheses and all subsequent occurrences only print the abbreviation.
  \item{\verb=\acp{}=}
    Same as \verb=\ac{}= but makes abbreviation plural.
  \item{\verb=\acf{}=}
    Same as first occurrence of \verb=\ac{}=.
  \item{\verb=\acl{}=}
    The long form of the acronym.
\end{description}


\section{Example}
\begin{itemize}
    \item \verb=\ac{PRF}= first occurrence: \ac{PRF}
    \item \verb=\ac{PRF}= second occurrence: \ac{PRF}
    \item \verb=\acp{PRF}=: \acp{PRF}
    \item \verb=\acl{PRF}=: \acl{PRF}
    \item \verb=\acf{PRF}=: \acf{PRF}
\end{itemize}

\bibliographystyle{abbrv}
\bibliography{main}
\end{document}

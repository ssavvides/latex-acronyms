\documentclass{article}
\usepackage{acro}
\usepackage{hyperref}
\usepackage{url}

% Crypto
\DeclareAcronym{PRN} {
	short = PRN,
	long = pseudorandom number
}
\DeclareAcronym{PRF} {
	short = PRF,
	long = pseudorandom function
}
\DeclareAcronym{PPT} {
	short = PPT,
	long = probabilistic polynomial-time
}
\DeclareAcronym{CPA} {
	short = CPA,
	long = chosen plaintext attack
}

% Security
\DeclareAcronym{TEE} {
	short = TEE,
	long = trusted execution environment
}
\DeclareAcronym{TCB} {
	short = TCB,
	long = trusted computing base
}
\DeclareAcronym{SGX} {
	short = SGX,
	long = software guard extensions
}
\DeclareAcronym{HbC} {
	short = HbC,
	long = Honest but Curious
}

% Systems
\DeclareAcronym{UDF} {
	short = UDF,
	long = user-defined function
}
\DeclareAcronym{IoT} {
	short = IoT,
	long = Internet of Things
}
\DeclareAcronym{VM} {
	short = VM,
	long = virtual machine
}
\DeclareAcronym{JVM} {
	short = JVM,
	long = Java Virtual Machine
}
\DeclareAcronym{tps} {
	short = tps,
	long = transactions per second
}

% Database Systems
\DeclareAcronym{DB} {
	short = DB,
	long = database
}
\DeclareAcronym{SQL} {
	short = SQL,
	long = structured query language
}
\DeclareAcronym{OLAP} {
	short = OLAP,
	long = online analytical processing
}
\DeclareAcronym{OLTP} {
	short = OLTP,
	long = online transactional processing
}

% Distributed Systems
\DeclareAcronym{RDD} {
	short = RDD,
	long = resilient distributed dataset
}
\DeclareAcronym{BFT} {
	short = BFT,
	long = byzantine fault-tolerant
}

% Algorithms
\DeclareAcronym{DAG} {
	short = DAG,
	long = directed acyclic graph
}

% Homomorphic Encryption
\DeclareAcronym{EDB} {
	short = EDB,
	long = Encrypted Database
}
\DeclareAcronym{FHE} {
	short = FHE,
	long = Fully Homomorphic Encryption
}
\DeclareAcronym{PHE} {
	short = PHE,
	long = Partially Homomorphic Encryption
}
\DeclareAcronym{AHE} {
	short = AHE,
	long = Additive Homomorphic Encryption
}
\DeclareAcronym{MHE} {
	short = MHE,
	long = Multiplicative Homomorphic Encryption
}
\DeclareAcronym{PPE} {
	short = PPE,
	long = Property-Preserving Encryption
}
\DeclareAcronym{OPE} {
	short = OPE,
	long = Order-Preserving Encryption
}
\DeclareAcronym{ORE} {
	short = ORE,
	long = Order-Revealing Encryption
}
\DeclareAcronym{FPE} {
	short = FPE,
	long = Format-Preserving Encryption
}
\DeclareAcronym{SRCH} {
	short = SRCH,
	long = searchable encryption
}
\DeclareAcronym{DET} {
	short = DET,
	long = deterministic
}
\DeclareAcronym{SDT} {
	short = SDT,
	long = Secure Data Type
}
\DeclareAcronym{SAHE} {
	short = SAHE,
	long = Symmetric Additive Homomorphic Encryption
}
\DeclareAcronym{SMHE} {
	short = SMHE,
	long = Symmetric Multiplicative Homomorphic Encryption
}
\DeclareAcronym{ASHE} {
	short = ASHE,
	long = Additively Symmetric Homomorphic Encryption
}

% Blockchains
\DeclareAcronym{PoW} {
	short = PoW,
	long = proof-of-work
}
\DeclareAcronym{PoS} {
	short = PoS,
	long = proof-of-stake
}


\begin{document}
\title{A personal collection of acronyms using \\ the ``acro'' package}
\author{Savvas Savvides}
\maketitle

\section{How to use}
All acronyms are defined in a file called ``acronyms.tex''. This document shows the most common commands of the acro package. For more information please refer to the official documentation of the Acro package~\cite{acro}.

Make sure you include the file containing the acronyms using the command: \verb=% Crypto
\DeclareAcronym{PRN} {
	short = PRN,
	long = pseudorandom number
}
\DeclareAcronym{PRF} {
	short = PRF,
	long = pseudorandom function
}
\DeclareAcronym{PPT} {
	short = PPT,
	long = probabilistic polynomial-time
}
\DeclareAcronym{CPA} {
	short = CPA,
	long = chosen plaintext attack
}

% Security
\DeclareAcronym{TEE} {
	short = TEE,
	long = trusted execution environment
}
\DeclareAcronym{TCB} {
	short = TCB,
	long = trusted computing base
}
\DeclareAcronym{SGX} {
	short = SGX,
	long = software guard extensions
}
\DeclareAcronym{HbC} {
	short = HbC,
	long = Honest but Curious
}

% Systems
\DeclareAcronym{UDF} {
	short = UDF,
	long = user-defined function
}
\DeclareAcronym{IoT} {
	short = IoT,
	long = Internet of Things
}
\DeclareAcronym{VM} {
	short = VM,
	long = virtual machine
}
\DeclareAcronym{JVM} {
	short = JVM,
	long = Java Virtual Machine
}
\DeclareAcronym{tps} {
	short = tps,
	long = transactions per second
}

% Database Systems
\DeclareAcronym{DB} {
	short = DB,
	long = database
}
\DeclareAcronym{SQL} {
	short = SQL,
	long = structured query language
}
\DeclareAcronym{OLAP} {
	short = OLAP,
	long = online analytical processing
}
\DeclareAcronym{OLTP} {
	short = OLTP,
	long = online transactional processing
}

% Distributed Systems
\DeclareAcronym{RDD} {
	short = RDD,
	long = resilient distributed dataset
}
\DeclareAcronym{BFT} {
	short = BFT,
	long = byzantine fault-tolerant
}

% Algorithms
\DeclareAcronym{DAG} {
	short = DAG,
	long = directed acyclic graph
}

% Homomorphic Encryption
\DeclareAcronym{EDB} {
	short = EDB,
	long = Encrypted Database
}
\DeclareAcronym{FHE} {
	short = FHE,
	long = Fully Homomorphic Encryption
}
\DeclareAcronym{PHE} {
	short = PHE,
	long = Partially Homomorphic Encryption
}
\DeclareAcronym{AHE} {
	short = AHE,
	long = Additive Homomorphic Encryption
}
\DeclareAcronym{MHE} {
	short = MHE,
	long = Multiplicative Homomorphic Encryption
}
\DeclareAcronym{PPE} {
	short = PPE,
	long = Property-Preserving Encryption
}
\DeclareAcronym{OPE} {
	short = OPE,
	long = Order-Preserving Encryption
}
\DeclareAcronym{ORE} {
	short = ORE,
	long = Order-Revealing Encryption
}
\DeclareAcronym{FPE} {
	short = FPE,
	long = Format-Preserving Encryption
}
\DeclareAcronym{SRCH} {
	short = SRCH,
	long = searchable encryption
}
\DeclareAcronym{DET} {
	short = DET,
	long = deterministic
}
\DeclareAcronym{SDT} {
	short = SDT,
	long = Secure Data Type
}
\DeclareAcronym{SAHE} {
	short = SAHE,
	long = Symmetric Additive Homomorphic Encryption
}
\DeclareAcronym{SMHE} {
	short = SMHE,
	long = Symmetric Multiplicative Homomorphic Encryption
}
\DeclareAcronym{ASHE} {
	short = ASHE,
	long = Additively Symmetric Homomorphic Encryption
}

% Blockchains
\DeclareAcronym{PoW} {
	short = PoW,
	long = proof-of-work
}
\DeclareAcronym{PoS} {
	short = PoS,
	long = proof-of-stake
}
=. The most commonly used commands of are:
\begin{description}
  \item{\verb=\acresetall=}
    Resets all acronyms.
  \item{\verb=\ac{}=}
    The most common command. The first occurrence prints the full acronym and its abbreviation in parentheses and all subsequent occurrences only print the abbreviation.
  \item{\verb=\acp{}=}
    Same as \verb=\ac{}= but makes abbreviation plural.
  \item{\verb=\acf{}=}
    Same as first occurrence of \verb=\ac{}=.
  \item{\verb=\acl{}=}
    The long form of the acronym.
\end{description}


\section{Example}
\begin{itemize}
    \item \verb=\ac{PRF}= first occurrence: \ac{PRF}
    \item \verb=\ac{PRF}= second occurrence: \ac{PRF}
    \item \verb=\acp{PRF}=: \acp{PRF}
    \item \verb=\acl{PRF}=: \acl{PRF}
    \item \verb=\acf{PRF}=: \acf{PRF}
\end{itemize}

\bibliographystyle{abbrv}
\bibliography{main}
\end{document}
